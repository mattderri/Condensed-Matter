\documentclass[10.75pt,a4paper,openright,bottom=2cm]{article}
\usepackage[english]{babel}
\usepackage[T1]{fontenc} 
\usepackage[utf8]{inputenc}
\usepackage{graphicx}
\usepackage{auto-pst-pdf} 
\usepackage{float}
\usepackage{graphicx}
\usepackage{wrapfig}
%\usepackage{subcaption}
\usepackage{textcomp}
\usepackage{geometry}
\usepackage{pdfpages}
\usepackage{amsmath}
\usepackage{amsfonts}
\usepackage{wrapfig}
\usepackage{lipsum} 
\usepackage{fancyhdr}
\usepackage{amsmath}
\usepackage{graphicx}
\usepackage{tcolorbox}
\usepackage{bbm}
\usepackage{braket}
\usepackage{amssymb}
\usepackage{pifont}
\newcommand{\cmark}{\ding{51}}%
\newcommand{\xmark}{\ding{55}}%
\usepackage[table]{xcolor, colortbl}
\usepackage{cancel}
\DeclareMathAlphabet{\pazocal}{OMS}{zplm}{m}{n}
\usepackage[colorlinks=true, allcolors=blue]{hyperref}
\usepackage{multicol}
\usepackage{physics}
\usepackage[super]{nth}
\usepackage{bbm}
\usepackage{tikz}
%\usepackage[compat=1.1.0]{tikz-feynman}
\usetikzlibrary{positioning}
\usepackage{circuitikz}
\usetikzlibrary{arrows,shapes,positioning}
\usetikzlibrary {arrows.meta} 
\usetikzlibrary{angles,quotes}
\usepackage{subfig}
\usepackage{mathtools}
\renewcommand{\Vec}[1]{\boldsymbol{#1}}
% \newcommand{\beginbox}[1]{\begin{tcolorbox}[width=\textwidth,colback={yellow!50},title={#1},colbacktitle={gray!50},coltitle=black]}
% \renewcommand{\endbox}{\end{tcolorbox}\noindent}
% \begin{tcolorbox}[width=\textwidth,colback={yellow!50},title={Rosenbluth Cross Section},colbacktitle={gray!50},coltitle=black]
\title{Condensed Matter :(}
\author{Matteo D'Errigo}

\begin{document}
\maketitle
\tableofcontents
% \begin{abstract}
% \end{abstract}
\newpage
\section{Crystal Lattices}
The \textbf{Bravais lattice} specifies the periodic array in which the specific structure of the crystal are arranged. There are two equivalent definitions:
\begin{enumerate}
    \item A Bravais lattice is an infinite array of discrete points with an arrangement and orientation that appears exactly the same from whichever of the points the array is viewed
    \item A 3D Bravais lattice consists of all points with position vectors $\Vec{R}$ of the form:
    \begin{equation}
    \label{R}
    \Vec{R}=n_1\Vec{a_1}+n_2\Vec{a_2}+n_3\Vec{a_3}
    \end{equation}
    where $\Vec{a_1}, \Vec{a_2}$ and $\Vec{a_3}$ are any three vectors not all in the same plane, $n_1, n_2$ and $n_3$ range through all integer values.
\end{enumerate}
The vectors $\Vec{a_i}$ are called \textbf{primitive vectors} and they generate or span the lattice. The two definitions are equivalent, any array satisfying 1. satisfies 2. as well.\\
A volume of space that, when translated through all the vectors in a Bravais lattice, just fills all of space without either overlapping itself or leaving voids is called a \textbf{primitive unit cell}. There is no unique way of choosing a primitive cell for a given Bravais lattice. A primitive cell must contain exactly one lattice point: it follows that if $n$ is the density of points in the lattice and $v$ is the volume of the cell, then $nv=1$. Therefore, $v=1/n$ and this result holds for any primitive cell. The obvious primitive cell to associate with a particular set of primitive vectors $\{\Vec{a_1},\Vec{a_2},\Vec{a_3}\}$ is the set of all points $\Vec{r}$ of the form:
\[
\Vec{r}=x_1\Vec{a_1}+x_2\Vec{a_2}+x_3\Vec{a_3}
\]
for $x_i$ ranging continuously between 0 and 1. This choice has the disadvantage of not displaying the full symmetry of the lattice. It is important to work with cells that have the full symmetry of their lattice and there are two solutions to this problem:
\begin{itemize}
    \item \textbf{Conventional unit cell}: one can fill up space with non-primitive unit cells, the conventional unit cell is generally chosen to be bigger than the primitive cell and to have the required symmetry
    \item \textbf{Wigner-Seitz primitive cell}: one can choose a primitive cell with the full symmetry of the lattice. The most common choice is the Wigner-Seitz cell which is defined as the region of space that is closer to a specific point than to any other lattice point. Since there is nothing in this definition that refers to any particular choice of primitive vectors, the Wigner-Seitz cell will be as symmetrical as the Bravais lattice. 
\end{itemize}
A \textbf{crystal structure} consists of identical copies of the same physical unit, called \textbf{basis}, located at all the points of a Bravais lattice. 
\hline
\begin{itemize}
    \item Simple cubic (SC): it can be spanned by three mutually perpendicular primitive vectors of equal length
    \item Body-centered cubic (BCC): it is formed by adding to the SC an additional point at the center of each little cube. If the original SC lattice is generated by primitive vectors:
    \[
    a\Vec{\hat{x}} \quad a\Vec{\hat{y}} \quad a\Vec{\hat{z}}
    \]
    then a set of primitive vectors for the BCC could be:
    \[
    \Vec{a_1}=a\Vec{\hat{x}} \quad \Vec{a_2}=a\Vec{\hat{y}} \quad \Vec{a_3}=\frac{a}{2}(\Vec{\hat{x}}+\Vec{\hat{y}}+\Vec{\hat{z}})
    \xleftrightarrow[]{}
    \Vec{a_1}=\frac{a}{2}(\Vec{\hat{y}}+\Vec{\hat{z}}-\Vec{\hat{x}}) \quad \Vec{a_2}=\frac{a}{2}(\Vec{\hat{z}}+\Vec{\hat{x}}-\Vec{\hat{y}}) \quad \Vec{a_3}=\frac{a}{2}(\Vec{\hat{x}}+\Vec{\hat{y}}+\Vec{\hat{z}})
    \]
    \item Face-centered cubic (FCC): it is constructed by adding to the SC an additional point in the center of each square face. A symmetric set of primitive vectors for the FCC is:
    \[
    \Vec{a_1}=\frac{a}{2}(\Vec{\hat{y}}+\Vec{\hat{z}}) \quad \Vec{a_2}=\frac{a}{2}(\Vec{\hat{z}}+\Vec{\hat{x}}) \quad \Vec{a_3}=\frac{a}{2}(\Vec{\hat{x}}+\Vec{\hat{y}})
    \]
\end{itemize}
\section{The Reciprocal Lattice}
Consider a set of points $\Vec{R}$ constituting a Bravais lattice and a plane wave $e^{i\Vec{k}\cdot\Vec{r}}$. For general $\Vec{k}$ such a plan wave will not have the periodicity of the Bravais lattice but it will for certain special choices. The set of all wave vectors $\Vec{K}$ that yield plane waves with the periodicity of a given Bravais lattice is known as its \textbf{reciprocal lattice}. This is true if:
\[
\exp{i\Vec{K}\cdot(\Vec{r}+\Vec{R})}=\exp{i\Vec{K}\cdot\Vec{r}}\xleftrightarrow{}\exp{i\Vec{K}\cdot\Vec{R}}=1\Rightarrow\Vec{K}\cdot\Vec{R}=2\pi m
\]
for all $\Vec{R}$ in the Bravais lattice. Note that a reciprocal lattice is defined with reference to a particular Bravais lattice, referred as the direct lattice.\\
Let $\Vec{a_1},\Vec{a_2}$ and $\Vec{a_3}$ be a set of primitive vectors for the direct lattice, the reciprocal lattice can be generated by the following three primitive vectors:
\[
\Vec{b_1}=2\pi\frac{\Vec{a_2}\times\Vec{a_3}}{\Vec{a_1}\cdot(\Vec{a_2}\times\Vec{a_3})} \quad
\Vec{b_2}=2\pi\frac{\Vec{a_3}\times\Vec{a_1}}{\Vec{a_1}\cdot(\Vec{a_2}\times\Vec{a_3})}\quad
\Vec{b_3}=2\pi\frac{\Vec{a_1}\times\Vec{a_2}}{\Vec{a_1}\cdot(\Vec{a_2}\times\Vec{a_3})}\quad
\]
To verify this, note that the vectors $\Vec{b_i}$ satisfy $\Vec{b_i}\cdot\Vec{a_j}=2\pi\delta_{ij}$. Any vector $\Vec{k}$ can be written as:
\[
\Vec{k}=k_1\Vec{b_1}+k_2\Vec{b_2}+k_3\Vec{b_3}
\]
If $\Vec{R}$ is a lattice vector it is in the form of \hyperref[R]{Equation} \ref{R}, hence we have:
\[
\Vec{k}\cdot\Vec{R}=2\pi(k_1n_1+k_2n_2+k_3n_3)
\]
For $\exp{i\Vec{k}\cdot\Vec{R}}$ to be equal to 1 for all $\Vec{R}$, $\Vec{k}\cdot\Vec{R}$ must be $2\pi m$ for $m$ integer for any choices of integers $n_i$. This requires $k_i$ to be integers as well, $\Vec{K}$ is a reciprocal lattice vector for those vectors that are linear combinations of the $\Vec{b_i}$ for integer coefficients. It follows that the reciprocal lattice is a Bravais lattice and $\Vec{b_i}$ can be taken as primitive vectors.\\
There is a relation between vectors in the reciprocal lattice and planes of points in the direct lattice. Given a particular Bravais lattice, a \textbf{lattice plane} is defined to be any plane containing at least three non-collinear Bravais lattice points. A \textbf{family of lattice planes} is a set of parallel equally spaced lattice planes which together contain all the points of the 3D Bravais lattice.\\
\textit{For any family of lattice planes separated by a distance $d$, there are reciprocal lattice vectors perpendicular to the planes, the shortest of which have a length of $2\pi/d$. Conversely, for any reciprocal lattice vector $\Vec{K}$ there is a family of lattice planes normal to $\Vec{K}$ and separated by a distance $d$, where $2\pi/d$ is the length of the shortest reciprocal lattice vector parallel to $\Vec{K}$.}\\
The correspondence between reciprocal lattice vectors and families of lattice planes provides a convenient way to specify the orientation of a lattice plane. Generally, one describes the orientation of a plane by giving a vector normal to the plane. Since we know there are reciprocal lattice vectors normal to any family of lattice planes, it is natural to pick a reciprocal lattice vector to represent the normal. To make the choice unique, one uses the shortest one, arriving at the \textbf{Miller indices} of the plane.\\
\textit{The miller indices of a lattice plane are the coordinates of the shortest reciprocal lattice vector normal to that plane, with respect to a specified set of primitive reciprocal lattice vectors. A plane with Miller indices $k,k,l$ is normal to the reciprocal lattice vector $h\Vec{b_1}+k\Vec{b_2}+l\Vec{b_3}$}\\
The Miller indices are integers, since the normal to the plane is specified by the shortest perpendicular reciprocal lattice vector, the integers $h,k,l$ can have no common factor. Lattice planes are specified by giving the Miller indices in parentheses $(h,k,l)$. Commas are eliminated by writing $\overline{n}$ instead of $-n$. To specify directions in the direct lattice, but to avoid confusion, square brackets are used instead of parentheses. The $(100), (010)$ and $(001)$ planes are all equivalent in a cubic crystal, they are referred collectively as the $\{100\}$ planes. Similarly for the directions, one write $\Braket{100}$.
\section{Diffraction}
When waves propagates through a crystal with wavelength of the same order of the primitive cell dimensions, there will be the phenomenon of diffraction. Different type of waves can propagates through the crystal:
\begin{itemize}
    \item Electromagnetic waves (X-rays)
    \item Charged particles (electrons)
    \item Neutral particles (neutrons)
\end{itemize}
For X-rays, diffraction is caused by the electrons in the crystal, for charged particles it is due to electrons and nuclei while for neutral particles is produced essentially by the nuclei in the crystal.\\
Typical interatomic distances in a solid are on the order of an angstrom $(10^{-8}$\,cm). An electromagnetic probe of the microscopic structure must have a wavelength at least this short, corresponding to an energy of order of $\sim10^3$\,eV: these are characteristic X-ray energies.\\
In every scattering phenomenon from a point $\Vec{\rho}$, the amplitude of the diffused wave in the point $\Vec{r}$ is proportional to the amplitude of the incoming wave in the scattering point (which can be approximated as a plane wave being the source far away) and to the amplitude of the outgoing spherical wave:
\[
\text{Incoming plane wave:}\;\; F_0\exp{i\Vec{k}\cdot\Vec{\rho}} \quad \text{Outgoing spherical wave:}\;\; D\frac{\exp{i\Vec{k}\cdot\Vec{r}}}{\Vec{r}}
\]
In the case X-rays are used, the points $\Vec{\rho}$ indicate the position of all the electrons in the crystal, which are the only scattering centers being much lighter than the nuclei. The proportional factor $D$ is given by:
\[
D=\frac{e^2}{mc^2}\sqrt{\frac{1+\cos^2\phi}{2}}
\]
where $\phi$ is the incident angle between the incoming direction and the diffusion one.\\
Scattering from different centers in different points gives place to interference which will be constructive only in specific directions, depending on the crystal periodicity and the wavelength. Consider now the scattering of plane waves from a crystal point $\Vec{\rho}$ towards an external point defined by $\Vec{R}$ from the origin. Using the following approximations:
\[
\Vec{R}=\Vec{r}+\Vec{\rho} \quad \Vec{R},\Vec{r}\gg\Vec{\rho} \quad \Vec{r}\simeq\Vec{R}-\Vec{\rho}\cos\theta
\]
the amplitude of the scattered wave then becomes:
\[
A\propto\frac{\exp{i\Vec{k}\cdot\Vec{\rho}}\exp{i\Vec{k}\cdot(\Vec{R}-\Vec{\rho}\cos\theta)}}{\Vec{R}}
\]
The factor $\exp{i\Vec{k}\cdot\Vec{R}}$ does not depend on the diffusion point, hence it can be treated as a constant. The factor $\exp{-i\Vec{k}\Vec{\rho}\cos\theta}$ can be written as:
\[
\exp{-i\Vec{k}\Vec{\rho}\cos\theta}\simeq\exp{-i\Vec{\rho}\left(\Vec{k}\frac{\Vec{R}}{|\Vec{R}|}\right)}=\exp{-i\Vec{\rho}\cdot\Vec{k}'} \quad \text{where}\;\;\Vec{k}'=\Vec{k}\frac{\Vec{R}}{|\Vec{R}|}
\]
$\Vec{k}'$ defines the direction of the scattered wave. It follows that we have an amplitude of the type:
\[
A\propto\exp{i\Vec{k}\cdot\Vec{\rho}}\exp{-i\Vec{k}'\cdot\Vec{\rho}}=\exp{-i\Vec{\rho}\cdot(\Vec{k}'-\Vec{k})}=\exp{i\Delta\Vec{k}\cdot\Vec{\rho}} \quad \Delta\Vec{k}=\Vec{k}'-\Vec{k}
\]
Considering all the terms we have:
\[
A=\frac{e^2}{mc^2}\sqrt{\frac{1+\cos^2\phi}{2}}\frac{F_0}{\Vec{R}}\exp{i\Vec{k}\cdot\Vec{R}}\exp{i\Delta\Vec{k}\cdot\Vec{\rho}}=C(\phi)\exp{i\Delta\Vec{k}\cdot\Vec{\rho}}
\]
The wave vector of the scattered wave $\Vec{k}'$ and the wave vector of the incoming wave $\Vec{k}$ must have equal modulus because we are considering elastic scattering.\\
If the points $\Vec{\rho}$ belong to a crystal lattice, it is possible to relate them to the points $\Vec{\rho_0}$ of the primitive cell:
\[
\Vec{\rho}=\Vec{\rho_0}+n_1\Vec{t_1}+n_2\Vec{t_2}+n_3\Vec{t_3}
\]
The relative total contribution will then be proportional to:
\begin{equation}
\label{rhozero}
\exp{-i\Vec{\rho_0}\cdot\Delta\Vec{k}}\sum_{n_1n_2n_3}\exp{-i(n_1\Vec{t_1}+n_2\Vec{t_2}+n_3\Vec{t_3})\cdot\Delta\Vec{k}}
\end{equation}
Summing over the translations will give e meaningful contribution only if the terms are equal to 1, i.e.:
\[
\exp{-i\Vec{t_n}\cdot\Delta\Vec{k}}=1
\]
Using the definition of reciprocal lattice, this can be written as $\Delta\Vec{k}=\Vec{h}$ where $\Vec{h}$ is a generic vector of the reciprocal lattice. This is the \textbf{Laue law}.\\
It is possible to show that when the number of lattice points is large enough, there will be scattering of noticeable intensity only in the directions defined by the Laue condition. The scattered intensity is proportional to the product of three terms of the form:
\[
\left|\sum_{n=0}^{N-1}\exp{-in\Vec{t}\cdot\Delta\Vec{K}}\right|^2=\left|\frac{1-\exp{-iN\Vec{t}\cdot\Delta\Vec{k}}}{1-\exp{-i\Vec{t}\cdot\Delta\Vec{k}}}\right|^2=\frac{\sin^2(N\Vec{t}\cdot\Delta\Vec{k}/2)}{\sin^2(\Vec{t}\cdot\Delta\Vec{k}/2)}
\]
Functions of this type are oscillating function of $\Vec{t}\cdot\Vec{k}$, with a maximum in $(2n+1)\pi/N$ and a minimum corresponding to zero in $2n\pi/N$. Taking into account the product of three factors of that type, the scattered intensity is different from zero only around those points where the function has a maximum value. The maximum intensity condition can be written as:
\[
\Vec{t_1}\cdot\Delta\Vec{k}=2\pi n \quad \Vec{t_2}\cdot\Delta\Vec{k}=2\pi m \quad \Vec{t_3}\cdot\Delta\Vec{k}=2\pi r
\]
where $n,m$ and $r$ are integer numbers. These are the Laue equations for lattice diffraction.\\
So far, we have not considered the structure of the primitive cell but only the effect of translation symmetry. To get the total amplitude, the contribution from \hyperref[rhozero]{Equation \ref{rhozero}} have to be summed over all the points $\Vec{\rho}$ and take into account the density of scattering particles. In the case of a large number of elementary cells, it is possible to use the approximation:
\[
\frac{\sin^2(Nx/2)}{\sin^2(x/2)}\simeq2\pi N\delta(x)
\]
In our case, this gives us:
\[
I=\prod_{i=1}^3\left|\sum_{n_i=0}^{N_i-1}\exp{-in_i\Vec{t_i}\cdot\Delta\Vec{K}}\right|^2=\prod_{i=1}^3\frac{\sin^2(N_i\Vec{t_i}\cdot\Delta\Vec{k}/2)}{\sin^2(\Vec{t_i}\cdot\Delta\Vec{k}/2)}\simeq(2\pi)^3N_1N_2N_3\delta(\Delta\Vec{k}-\Vec{G})
\]
The intensity coming from the $\Vec{\rho_0}$ is given by:
\[
I=\left|\int_\Omega d\Vec{\rho_0}n(\Vec{\rho_0})\exp{-i\Vec{\rho_0}\cdot\Delta\Vec{k}}\right|^2
\]
$n(\Vec{\rho_0})$ is the density of scattered particles in the elementary cell. Putting now everything together we obtain:
\[
I=C^2(\phi)(2\pi)^3N\left|\int_\Omega d\Vec{\rho_0}n(\Vec{\rho_0})\exp{-i\Vec{\rho_0}\cdot\Delta\Vec{k}}\right|^2\delta(\Delta\Vec{k}-\Vec{G})
\]
where $N=N_1N_2N_3$ is the total number of cells. If there are more atoms in a cell and each atom contributes independently, then the elementary cell contribution can be splitted in:
\[
n(\Vec{\rho_0})=\sum_jn_j(\Vec{\rho_0}-\Vec{\rho_j})
\]
where $\Vec{\rho_j}$ is the position of each atom in the cell. If we substitute this in the expression for the intensity found above we obtain an additional factor:
\begin{align*}
\int_\Omega d\Vec{\rho_0}\sum_jn_j(\Vec{\rho_0}-\Vec{\rho_j})\exp{-i\Vec{\rho_0}\cdot\Vec{G}}&=\sum_j\exp{-i\Vec{G}\cdot\Vec{\rho_j}}\int_\Omega d\Vec{\rho_0}n_j(\Vec{\rho_0}-\Vec{\rho_j})\exp{-i\Vec{G}\cdot(\Vec{\rho_0}-\Vec{\rho_j})}\\
&=\sum_j\exp{-i\Vec{G}\cdot\Vec{\rho_j}}f_j(\Vec{G})
\end{align*}
If there are equal atoms in different points of the elementary cells, the factors $f_j(\Vec{G})$ are equal and the scattering amplitude is proportional to the \textbf{geometrical structure factor}:
\[
S(\Vec{G})=\sum_j\exp{-i\Vec{G}\cdot\Vec{\rho_j}}
\]
If the electronic distributions around the nuclei are spherical, by switching to polar coordinates we get:
\[
f_j(\Vec{G})=2\pi\int_0^\infty drr^2n_j(\Vec{r})\int_0^\pi d\theta\sin\theta\exp{-iGR\cos\theta}=4\pi\int_0^\infty drn_j(\Vec{r})r^2\frac{\sin(Gr)}{Gr}
\]
This will obviously depend on the distribution $n_j(r)$:
\[
n_j(r)=\frac{Z}{4\pi r^3/3}\to f_j(\Vec{G})=Z \qquad n_j(r)=\frac{\exp{-2r/a_0}}{\pi a_0^3}\to f_j(\Vec{G})=\frac{16}{(4\pi G^2a_0^2)^2}
\]
From Laue diffraction conditions it is possible to obtain the \textbf{Bragg law} by considering the family of parallel lattice planes associated to every reciprocal lattice vector. Take now a set of wave vectors of reciprocal lattice $\Vec{G}$ orthogonal to the family of planes $(hkl)$:
\[
\Vec{G}=h\Vec{b_1}+k\Vec{b_2}+l\Vec{b_3}
\]
Apply the Laue law to it:
\[
\Vec{k}'-\Vec{k}=\Vec{G}\to\cancel{|\Vec{k}'|^2}=\cancel{|\Vec{k}|^2}+|\Vec{G}|^2+2\Vec{k}\cdot\Vec{G}\Rightarrow\frac{|\Vec{G}|^2}{2}=-\Vec{k}\cdot\Vec{G}
\]
This can be written as:
\[
k\sin\theta=\frac{1}{2}G\to\frac{2\pi}{\lambda}\sin\theta=\frac{1}{2}\frac{2\pi n}{d}\Rightarrow2d\sin\theta=n\lambda
\]
This could have been obtained, without using reciprocal lattice vectors, by considering the path difference between the incoming and the scattered ray equal to $2d\sin\theta$, where $\theta$ is the angle of incidence. In order to have a constructive interference, this path difference must be an integral number of wavelengths:
\[
n\lambda=2d\sin\theta
\]
% The von Laue approach regards the crystal as composed of identical microscopic objects places at the sites $\Vec{R}$ of a Bravais lattice, each of which can radiate incident radiation in all directions. Sharp peaks are observed only in directions and at wavelengths for which the rays scattered from all lattice points interfere constructively.\\
% Consider now two scatterers separated by a displacement vector $\Vec{d}$. Let an X-ray be incident from very far away along a direction $\hat{n}$ with wavelength $\lambda$ and wave vector $\Vec{k}=2\pi\hat{n}/\lambda$. A scattered ray in the direction $\hat{n}'$ with wavelength $\lambda$ and wave vector $\Vec{k}'=2\pi\hat{n}'/\lambda$. The path difference between the two rays is:
% \[
% \Vec{d}\cdot(\hat{n}-\hat{n}')=d\cos\theta+d\cos\theta'
% \]
% The condition for constructive interference is:
% \[
% \Vec{d}\cdot(\hat{n}-\hat{n}')=m\lambda\Rightarrow\Vec{d}\cdot(\Vec{k}-\Vec{k}')=2\pi m
% \]
% for integer values of $m$. Considering now an array of scatterers at the sites of a Bravais lattice, the condition that all scattered rays interfere constructively is that the condition above holds true for all values of $\Vec{d}$ that are Bravais lattice vectors:
% \[
% \Vec{R}\cdot(\Vec{k}-\Vec{k}')=2\pi m\Rightarrow\exp{i(\Vec{k}'-\Vec{k}\cdot\Vec{R})}=1
% \]
% for all Bravais lattice vectors $\Vec{R}$ and integer values of $m$. Comparing this with the definition of reciprocal lattice, it is possible to conclude that the Laue condition states that \textit{constructive interference will occur provided that the change in wave vector $\Vec{k}'-\Vec{k}$ is a vector of the reciprocal lattice.}\\
% This discussion is based on the condition that rays scattered from each primitive cell should interfere constructively. If the crystal structure is that of a monoatomic lattice with an $n-$atom basis, the contents of each primitive cell can be further analysed into a set of identical scatterers at position $\Vec{d_1},\cdots,\Vec{d_n}$ within the cell. If the Bragg peak is associated with a change in wave vector, then the phase difference between the rays scattered at $\Vec{d_i}$ and $\Vec{d_j}$ will be $\Vec{K}\cdot(\Vec{d_i}-\Vec{d_j})$ and the amplitude of the two rays will differ by a factor $\exp{i\Vec{K}\cdot(\Vec{d_i}-\Vec{d_j})}$. The amplitudes of the rays scattered at $\Vec{d_1},\cdots,\Vec{d_n}$ are in the ratio $\exp{i\Vec{K}\cdot\Vec{d_1}},\cdots,\exp{i\Vec{K}\cdot\Vec{d_n}}$. The net ray scattered by the entire primitive cell is the sum of the individual rays, with an amplitude containing the factor:
% \[
% S_{\Vec{K}}=\sum_{j=1}^n\exp{i\Vec{K}\cdot\Vec{d_j}}
% \]
% This is known as \textbf{geometrical structure factor}. The intensity in the Bragg peak, proportional to the absolute value of the amplitude, will contain a factor $|S_{\Vec{K}}|^2$.\\
% If the ions in the basis are not identical, the structure factor assumes the form:
% \[
% S_{\Vec{K}}=\sum_{j=1}^nf_j(\Vec{K})\exp{i\Vec{K}\cdot\Vec{d_j}}
% \]
% where $f_j$ is known as the \textbf{atomic form factor} and it is entirely determined by the internal structure of the ion that occupies position $\Vec{d_j}$ in the basis. Identical ions have identical form factors, so in the monoatomic case this reduces back to the geometrical structure factor multiplied by the common value of the form factors.
\section{Lattice Dynamics of Crystals}
[...buco di svariate (credo 2) lezioni...]\\
In a system of many particles interacting via electromagnetic forces, the general problem is to solve the Schr\"odinger equation:
\[
H\Psi=E\Psi
\]
where both the Hamiltonian both the wave function depend on the coordinates of all the particles. The Hamiltonian $H$ can be seen as the sum of different terms:
\[
H=T_e+T_N+V_{NN}+V_{ee}+V_{eN}+C_e^{\text{rel}}
\]
The first one is the kinetic term for the electrons:
\[
T_e=\sum_e\frac{p_e^2}{2m}=-\sum_i\frac{\hbar^2}{2m}\nabla_i^2
\]
where $i$ denotes the coordinates of the electronic positions. The second term regards the kinetic part of the nuclei:
\[
T_N=\sum_N\frac{p_N^2}{2M}=-\sum_I\frac{\hbar^2}{2M}\nabla^2_I
\]
where $I$ denotes the coordinates of the nuclear positions. Then there are the potential terms:
\[
\left\{
\begin{aligned}
&\text{Nuclei interaction:} &&V_{NN}=\frac{1}{2}\sum_{I\neq J}\frac{Z_IZ_Je^2}{4\pi\varepsilon_0|\Vec{R_I}-\Vec{R_J}|}\\
&\text{Electrons interaction:} &&V_{ee}=\frac{1}{2}\sum_{i\neq j}\frac{e^2}{4\pi\varepsilon_0|\Vec{r_i}-\Vec{r_j}|}\\
&\text{Electron-nucleus interaction:} &&V_{eN}=-\sum_{i,J}\frac{Z_Je^2}{4\pi\varepsilon_0|\Vec{r_i}-\Vec{R_J}|}\\
\end{aligned}
\right.
\]
The last term accounts for relativistic corrections on electrons. The Schr\"odinger equation can be simplified by decomposing the wave function in a product of simpler functions:
\[
\Psi_{nv}(\Vec{r},\Vec{R})=F_{n,v}(\Vec{R})\psi_n(\Vec{r},\Vec{R})
\]
$\Vec{r}$ denotes the electronic coordinates while $\Vec{R}$ the nuclear ones. We can now use this in the Schr\"odinger equation and see what are the contributions coming from the various terms. The electronic kinetic term gives us:
\[
T_e\Psi_{nv}(\Vec{r},\Vec{R})=-\frac{\hbar^2}{2m}\nabla_r^2\Psi_{nv}(\Vec{r},\Vec{R})=-\frac{\hbar^2}{2m}F_{n,v}(\Vec{R})\nabla_r^2\psi_n(\Vec{r},\Vec{R})
\]
For the nuclear kinetic term instead one obtains:
\begin{align*}
T_N\Psi_{nv}(\Vec{r},\Vec{R})&=-\frac{\hbar^2}{2M}\nabla_R^2\Psi_{nv}(\Vec{r},\Vec{R})\\
&=-\frac{\hbar^2}{2M}\left[\psi_n(\Vec{R},\Vec{r})\nabla_R^2F_{n,v}(\Vec{R})+\cancel{2\nabla_RF_{n,v}(\Vec{R})\nabla_R\psi_n(\Vec{r},\Vec{R})}+\cancel{F_{n,v}(\Vec{R})\nabla_R^2\psi_n(\Vec{r},\Vec{R})}\right]
\end{align*}
The last two terms give a contribution much smaller than the first one, hence they get neglected. The Schr\"odinger equation now becomes:
\begin{align*}
&-\frac{\hbar^2}{2m}F_{n,v}(\Vec{R})\nabla_r^2\psi_n(\Vec{r},\Vec{R})-\frac{\hbar^2}{2M}\psi_n(\Vec{R},\Vec{r})\nabla_R^2F_{n,v}(\Vec{R})+\\
&+(V_{ee}+V_{NN}+V_{eN}+C_e^{\text{rel}})F_{n,v}(\Vec{R})\psi_n(\Vec{r},\Vec{R})=EF_{n,v}(\Vec{R})\psi_n(\Vec{r},\Vec{R})
\end{align*}
Dividing everything by $F_{n,v}(\Vec{R})\psi_n(\Vec{r},\Vec{R})$ this gets simplified in:
\[
\underbrace{-\frac{\hbar^2}{2m}\frac{\nabla_r^2\psi_n(\Vec{r},\Vec{R})}{\psi_n(\Vec{r},\Vec{R})}+V_{ee}+V_{eN}+C_e^{\text{rel}}}_{=E_n(\Vec{R})}-\frac{\hbar^2}{2M}\frac{\nabla_R^2F_{n,v}(\Vec{R})}{F_{n,v}(\Vec{R})}+V_{NN}=E
\]
There is a situation of the type $f(x,y)+g(y)=$constant, hence it follows that $f(x,y)=f(y)$. In our case, it means that the first four terms must be independent on $\Vec{r}$ and form a function of $\Vec{R}$ only, denoted with $E_n(\Vec{R})$. We then have:
\[
-\frac{\hbar^2}{2M}\frac{\nabla_R^2F_{n,v}(\Vec{R})}{F_{n,v}(\Vec{R})}+[V_{NN}+E_n(\Vec{R}]F_{n,v}(\Vec{R})=EF_{n,v}(\Vec{R})
\]
The fundamental state will be given by the minimal energy configuration, i.e. the one corresponding to the minimum of the potential: $V_0:=V_{NN}+E_0(\Vec{R})$.\\
Consider now a 1D chain of lattice constant $a$ formed by a large number $N$ of atoms of mass $M$. Denote with $u_n$ the longitudinal displacement of the $n$-th atom from the equilibrium position $t_n=na$. Fix the nuclei in the positions $R_n=na+u_n$ and indicate with $E_0(\{u_n\})$ the total ground state energy of the electron-nuclear system. In the study of small oscillations, it is convenient to expand in the energy in powers of the displacement $u_n$:
\[
E_0(\{u_n\})=E_0(0)+\frac{1}{2}\sum_{nn'}\underbrace{\frac{\partial^2E_0}{\partial u_n\partial u_{n'}}\Bigr|_{\substack{u_n=u_{n'}=0}}}_{:=D_{nn'}}u_nu_{n'}+\cdots
\]
Truncating the expansion to quadratic terms is called \textbf{harmonic approximation}, the quantity denoted above with $D_{nn'}$ goes under the name of \textbf{interatomic force constants} and the matrix $D$ formed with these $D_{nn'}$ is the \textbf{dynamical matrix}. The force constants $D_{nn'}$ represents the proportionality coefficients connecting the forces acting on the nuclei with the displacements:
\[
F_n=-\frac{\partial E_0}{\partial u_n}=-\sum_{n'}D_{nn'}u_{n'}
\]
From the definition, it follows that the matrix $D$ is real and symmetric:
\[
D_{nn'}=D_{n'n}
\]
Moreover, the translational symmetry of the lattice requires that:
\[
D_{nn'}=D_{mm'} \quad \text{if}\;\; t_n-t_{n'}=t_m-t_{m'}
\]
And finally there is the so-called sum rule:
\[
\sum_{n'}D_{nn'}=0 \quad \forall n
\]
This is a consequence of the fact that the forces vanish when all nuclear displacements are zero or when they are all equal. Consider now the classical equation of motion for the $n$-th nucleus of mass $M$ in the position $R_n=na+u_n$ under the force $F_n$:
\[
M\Ddot{u}_n=-\sum_{n'=1}^ND_{nn'}u_{n'} \quad n=1,2,\cdots,N
\]
This set of differential equations can be solved by looking for periodic solutions of the form $u_n(t)=A\exp{i(qna-\omega t)}$ where the amplitudes $A$ of the displacements are the same for all sites and the phases are controlled by the Bloch theorem. Substituting this type of solution in the equation above gives us:
\[
-M\omega^2A=-\sum_{n'=1}^ND_{nn'}\exp{-iq(na-n'a')}A\Rightarrow M\omega^2(q)=D(q) \quad D(q)=\sum_{n'=1}^ND_{nn'}\exp{-iq(na-n'a')}
\]
The Fourier transform $D(q)$ of the force constant matrix elements does not depend on the specific value $n$ because of translational symmetry.\\
We now apply this analysis to the case of a linear chain of atoms with nearest neighbour interactions only. This condition means that the only force constants different from zero are $D_{nn}, D_{nn+1}$ and $D_{n-1n}$ and from the general properties of the force constants there is a unique independent parameter, $C$:
\[
D_{nn}=2C \qquad D_{nn+1}=D_{n-1n}=-C
\]
The classical equations of motion now become:
\[
M\Ddot{u}_n=-C(2u_n-u_{n-1}-u_{n+1})
\]
Performing the same substitution as before with $u_n(t)=A\exp{i(qna-\omega t)}$ one obtains:
\[
-M\omega^2=-C(2-e^{iqa}-e^{-iqa})=-4C\sin^2(qa/2)\Rightarrow\omega(q)=\sqrt{\frac{4C}{M}}|\sin(qa/2)|
\]
Notice that the spectrum of vibrational frequencies extends from zero to a cutoff frequency of $\omega_{\max}=\sqrt{4C/M}$. In the long wavelength limit, i.e. $qa\ll1$, the dispersion relation takes the form:
\[
\omega(q)\simeq\sqrt{\frac{C}{M}}aq=v_sq \quad qa\ll1
\]
The proportionality coefficient $v_s$ between phonon frequency and phonon wave number represents the velocity of the sound in the medium. The periodic nature of the sine function appearing in the expression of $\omega(q)$ is telling us that the whole information is contained inside the \nth{1} Brillouin zone, i.e. between $-\pi/a$ and $+\pi/a$.\\
The ratio between the amplitudes of two nearest neighbours is given by:
\[
\frac{u_{n+1}}{u_n}=\frac{A\exp{i(qna+qa-\omega t)}}{A\exp{i(qna-\omega t)}}=\exp{iqa}
\]
Suppose now to have $|q|>\pi/a$ outside of the \nth{1} Brillouin zone, so $q=q'+2\pi n/a$ with $q'\in[-\pi/a,+\pi/a]$. If we take now the ratio we get:
\[
\frac{u_{n+1}}{u_n}=\exp{iq'a}\exp{2\pi in}
\]
We get the same type of results for vectors outside of the \nth{1} Brillouin zone. It is also possible to compute the group velocity, defined as follows:
\[
v_g=\frac{d\omega}{dq}=\sqrt{\frac{a^2C}{M}}\cos(qa/2)
\]
which vanishes for $q=\pm\pi/a$, on the boundaries of the \nth{1} Brillouin zone. For this value of $q$, the amplitude for $u_n$ takes the form:
\[
u_n=A\exp{i(qna-\omega t)}=A(-1)^n\exp{-i\omega t}
\]
This is a stationary wave which can be obtained as a combination of progressive and regressive wave:
\[
A[\sin(kx-\omega t)+\sin(kx+\omega t)]=2A\sin(kx)\cos(\omega t)
\]
Consider now the dynamics of a diatomic linear chain of lattice constant $a_0$ with two atoms of mass $M_1$ and $M_2$ in the unit cell. In the equilibrium configuration, assume that the atoms of mass $M_1$ occupy the position $R_n^{(1)}=na_0$ while the atoms of mass $M_2$ occupy $R_n^{(2)}=(n+1/2)a_0$. $u_n$ indicates the displacements of the atoms of mass $M_1$ and $v_n$ the displacements of the atoms of mass $M_2$. Again, we work in the assumption that only nearest neighbours interact with two different elastic constants: $C_1$ if the interaction is between atoms of the same cell and $C_2$ if it is between atoms of adjacent cells. The equations of motion are:
\[
\left\{
\begin{aligned}
&M_1\Ddot{u}_n=-C_1(u_n-v_n)-C_2(u_n-v_{n-1})\\
&M_2\Ddot{v}_n=-C_1(v_n-u_n)-C_2(v_n-u_{n+1})
\end{aligned}
\right.
\]
The periodic solutions for $u_n$ and $v_n$ now takes the form:
\[
u_n(t)=A_u\exp{i(qna-\omega t)} \qquad v_n(t)=A_v\exp{i(qna-\omega t)}
\]
Substituting them in the equations above gives us:
\[
\left\{
\begin{aligned}
&-M_1\omega^2A_u=-C_1(A_u-A_v)-C_2(A_u-A_ve^{-iqa})\\
&-M_2\omega^2A_v=-C_1(A_v-A_u)-C_2(A_v-A_ue^{iqa})
\end{aligned}
\right.
\to
\left\{
\begin{aligned}
&A_u(-M_1\omega^2+C_1+C_2)=A_v(C_1+C_2e^{-iqa})\\
&A_v(-M_2\omega^2+C_1+C_2)=A_u(C_1+C_2e^{iqa})
\end{aligned}
\right.
\]
This system of equations have a non-trivial solution if the determinant of the coefficients $A_u$ and $A_v$ is zero.
\[
\det\left(\begin{array}{cc}
    C_1+C_2-M_1\omega^2 & C_1+C_2e^{iqa} \\
    C_1+C_2e^{-iqa} & C_1+C_2-M_2\omega^2
\end{array}\right)=0
\]
Imposing the determinant to be equal to 0 gives us a second degree equation in $\omega^2$ whose solutions are:
\[
\omega^2=\frac{(C_1+C_2)(M_1+M_2)}{2M_1M_2}\pm\sqrt{\frac{(C_1+C_2)^2(M_1+M_2)^2}{4(M_1M_2)^2}-\frac{4C_1C_2\sin^2(qa/2)}{M_1M_2}}
\]
The $(+)$ solution corresponds to the \textbf{optical branch} while the $(-)$ solution is the \textbf{acoustic branch}. In the case $C_1=C_2=C$ we get:
\[
\omega^2=C\left(\frac{1}{M_1}+\frac{1}{M_2}\right)\pm C\sqrt{\left(\frac{1}{M_1}+\frac{1}{M_2}\right)^2-\frac{4\sin^2(qa/2)}{M_1M_2}}
\]
A further simplification is given by the case in which $C_1=C_2=C$ and $M_1=M_2=M$:
\[
\omega_+=\sqrt{\frac{4C}{M}}\cos(qa/4) \qquad \omega_-=\sqrt{\frac{4C}{M}}\sin(qa/4)
\]
The amplitudes $A_u$ and $A_v$ satisfies the relation:
\[
\frac{A_u}{A_v}=\frac{C_1+C_2e^{iqa}}{C_1+C_2-M_1\omega^2}=\frac{C_1+C_2-M_2\omega^2}{C_1+C_2e^{iqa}}
\]
Consider now the case in which $M_1=M_2=M$ but $C_1\neq C_2$ and work in the long wavelength limit, i.e. $qa\ll1$.
\[
\omega^2=\frac{C_1+C_2}{M}\pm\frac{1}{M}\sqrt{C_1^2+C_2^2+2C_1C_2\cos(qa)}\simeq\frac{C_1+C_2}{M}\pm\frac{1}{M}\sqrt{C_1^2+C_2^2+2C_1C_2\left(1-\frac{q^2a^2}{2}+\mathcal{O}((qa)^4)\right)}
\]
The $(+)$ and $(-)$ solutions, i.e. respectively the optical and acoustic branch now become:
\[
\left\{
\begin{aligned}
&\text{Optical branch}: &&\omega\simeq\sqrt{\frac{2(C_1+C_2)}{M}}+\mathcal{O}((qa)^2) &&&\frac{A_u}{A_v}=-1 &&&v_g=0\\
&\text{Acoustic branch}: &&\omega\simeq\sqrt{\frac{C_1C_2}{2M(C_1+C_2)}}(qa) &&&\frac{A_u}{A_v}=+1 &&&v_g=\sqrt{\frac{C_1C_2}{8m(C_1+C_2)}}
\end{aligned}
\right.
\]
In the acoustic branch in the long wavelength limit, the atoms vibrate in phase and with the same amplitude and the frequency $\omega$ is proportional to the wave number $q$. In the optical branch, $A_u$ and $A_v$ have opposite signs and the group velocity vanishes: the two atoms in the unit cell move in opposite directions while the \textit{center of mass} of the unit cell remains fixed.\\
So far, we have considered the classical dynamics of a linear chain. Now we move to the quantum mechanical counterpart of the same problem. Working in the harmonic approximation and nearest neighbour interactions, the Hamiltonian of the linear chain becomes:
\begin{equation}
\label{H}
H=\sum_n\frac{p_n^2}{2M}+\frac{1}{2}C\sum_n(2u_n^2-u_nu_{n+1}-u_nu_{n+1})
\end{equation}
where $u_n$ and $p_n$ are the coordinate and the conjugate momentum of the nucleus at the $n$-th site. However, instead of working with them it is more convenient to perform a canonical transformation, with the final aim to put the Hamiltonian in the diagonal form.
\[
p_q=\frac{1}{\sqrt{N}}\sum_{t_n}p_n\exp{+iqt_n} \qquad u_q=\frac{1}{\sqrt{N}}\sum_{t_n}u_n\exp{-iqt_n}
\]
where $t_n=na$. The inverse transformations are given by:
\[
p_n=\frac{1}{\sqrt{N}}\sum_qp_q\exp{-iqt_n} \qquad u_n=\frac{1}{\sqrt{N}}\sum_qu_q\exp{+iqt_n}
\]
It is easy to see that these transformations are canonical, i.e. the commutation rules are preserved.
\[
[u_q,p_{q'}]=i\hbar\delta_{qq'} \quad [u_q,u_{q'}]=[p_q,p_{q'}]=0
\]
It is now possible to express the original Hamiltonian in terms of the new displacements and conjugated momenta.
\[
\left\{
\begin{aligned}
&\sum_np_n^2=\sum_n\frac{1}{N}\sum_q\sum_{q'}p_qp_{q'}\exp{i(q+q')t_n}=\sum_qp_qp_{-q}\\
&\sum_nu_n^2=\sum_n\frac{1}{N}\sum_q\sum_{q'}u_qu_{q'}\exp{i(q+q')t_n}=\sum_qu_qu_{-q}\\
&\sum_nu_nu_{n+1}=\sum_n\frac{1}{N}\sum_q\sum_{q'}p_qp_{q'}\exp{i(q+q')t_n+iq'a}=\sum_qp_qp_{-q}e^{-iqa}
\end{aligned}
\right.
\]
The Hamiltonian of \hyperref[H]{Equation \ref{H}} can be written as:
\begin{equation}
\label{H1}
H=\sum_q\frac{p_qp_{-q}}{2M}+\frac{1}{2}C\sum_qu_qu_{-q}(2-e^{-iqa}+e^{+iqa})=\sum_q\left[\frac{p_qp_{-q}}{2M}+\frac{1}{2}M\omega^2(q)u_qu_{-q}\right]
\end{equation}
where $\omega^2(q)$ is given by the following:
\[
\omega^2(q)=\frac{C}{M}(2-e^{-iqa}+e^{+iqa})=\frac{4C}{M}\sin^2(qa/2)
\]
This shows that the linear chain of $N$ coupled harmonic oscillators is equivalent to $N$ uncoupled normal modes.\\
It is useful to define another canonical transformation to creation and destruction operators, so we define:
\[
a_q=\sqrt{\frac{m\omega(q)}{2\hbar}}u_q+i\sqrt{\frac{1}{2M\hbar\omega(q)}}p_{-q} \qquad a_q^\dagger=\sqrt{\frac{m\omega(q)}{2\hbar}}u_{-q}-i\sqrt{\frac{1}{2M\hbar\omega(q)}}p_q
\]
The commutation rules are preserved:
\[
[a_q,a_{q'}^\dagger]=\delta_{qq'} \quad [a_q,a_{q'}]=[a_q^\dagger,a_{q'}^\dagger]=0
\]
The displacements and the conjugated momenta gets now written in the form:
\[
\left\{
\begin{aligned}
&u_n=\frac{1}{\sqrt{N}}\sum_q\overbrace{\sqrt{\frac{\hbar}{2M\omega(q)}}(a_q+a_{-q}^\dagger)}^{u_q}\exp{+iqt_n}\\
&p_n=\frac{1}{\sqrt{N}}\sum_q\underbrace{(-i)\sqrt{\frac{M\hbar\omega(q)}{2}}(a_{-q}-a_q^\dagger)}_{p_q}\exp{-iqt_n}
\end{aligned}
\right.
\]
The Hamiltonian of \hyperref[H1]{Equation \ref{H1}} can be expressed in second quantization form:
\[
H=\sum_q\hbar\omega(q)\left(a_q^\dagger a_q+\frac{1}{2}\right)
\]
which can be seen as the sum of the Hamiltonians of $N$ independent linear harmonic oscillators of frequency $\omega(q)$.\\
The lattice dynamics discussed before was the one of a 1D crystal, now we address the general problem of a 3D crystal. Consider a general 3D crystal with $N$ unit cells, translation vectors $\Vec{t_n}$ and a basis of atoms in the positions $\Vec{d}_1,\Vec{d}_2,\cdots,\Vec{d}_n$. The generic position vector is given by $\Vec{t}_n+\Vec{d}_\nu+\Vec{u}_{n\nu}$ and the energy of the electronic-nuclear system is denoted by $E_0(\{\Vec{u}_{n\nu}\})$. The expansion up to second order gives us:
\[
E_0(\{\Vec{u}_{n\nu}\})=E_0(0)+\frac{1}{2}\sum_{n\nu\alpha,n'\nu'\alpha'}D_{n\nu\alpha,n'\nu'\alpha'}u_{n\nu\alpha}u_{n'\nu'\alpha'}
\]
where $\alpha,\alpha'=x,y,z$; $\nu,\nu'=1,\cdots,\nu_b$ where $\nu_b$ is the number of atoms forming the basis of the unit cell and $n=1,\cdots,N$. Explicitly, the interatomic force constants are defined as:
\[
D_{n\nu\alpha,n'\nu'\alpha'}=\frac{\partial^2E_0}{\partial u_{n\nu\alpha}\partial u_{n'\nu'\alpha'}}\Bigr|_{\substack{0}}
\]
The matrix $D$ is called the \textbf{dynamical matrix of the crystal in real space} and it has some properties:
\begin{enumerate}
    \item It is real and symmetric $D_{n\nu\alpha,n'\nu'\alpha'}=D_{n'\nu'\alpha',n\nu\alpha}$
    \item Translation symmetry implies $D_{n\nu\alpha,n'\nu'\alpha'}=D_{m\nu\alpha,m'\nu'\alpha'}$ if $\Vec{t}_n-\Vec{t}_{n'}=\Vec{t}_m-\Vec{t}_{m'}$
    \item The sum rule tells us $\sum_{n'\nu'}=D_{n\nu\alpha,n'\nu'\alpha'}=0$
\end{enumerate}
In the harmonic approximation, the classical equations of motion read:
\[
M_\nu\Ddot{u}_{n\nu\alpha}=-\sum_{n'\nu'\alpha'}D_{n\nu\alpha,n'\nu'\alpha'}u_{n'\nu'\alpha'}
\]
We solve the set of differential equations looking for solutions in the form:
\[
\Vec{u}_{n\nu}(t)=\Vec{A}_\nu(\Vec{q},\omega)\exp{i(\Vec{q}\cdot\Vec{t}_n-\omega t)}
\]
Replacing this solution in the equation above gives us:
\[
-M_\nu\omega^2A_{\nu\alpha}=-\sum_{n'\nu'\alpha'}D_{n\nu\alpha,n'\nu'\alpha'}A_{\nu'\alpha'}\exp{-i\Vec{q}(\Vec{t}_n-\Vec{t}_{n'})}
\]
Non-trivial solutions are obtained by solving the following:
\[
\norm{D_{\nu\alpha,\nu'\alpha'}(\Vec{q})-M_\nu\omega^2\delta_{\alpha\alpha'}\delta_{\nu\nu'}}=0 \quad \text{where}\;\; D_{\nu\alpha,\nu'\alpha'}(\Vec{q})=\sum_{n'}D_{n\nu\alpha,n'\nu'\alpha'}\exp{-i\Vec{q}\cdot(\Vec{t}_n-\Vec{t}_{n'})}
\]
The matrix $D(\Vec{q})$ has dimension $3\nu_b$, the secular equation produces $3\nu_b$ eigenvalues, called phonons or normal modes. At every vector $\Vec{q}$ there are $3\nu_b$ normal modes, giving rise to $3\nu_b$ phonon branches as $\Vec{q}$ varies inside the \nth{1} Brillouin zone. Let $\omega(\Vec{q},p)$ be the frequency of the $p$-th normal mode of wave vector $\Vec{q}$ and $\Vec{A}_\nu(\Vec{q},p)$ the corresponding polarization vectors. A mode $\omega(\Vec{q},p)$ is called \textbf{longitudinal} in case the polarization vectors $\Vec{A}_\nu(\Vec{q},p)$ are parallel to $\Vec{q}$, while it is called \textbf{transverse} if they are perpendicular to it.
\end{document}